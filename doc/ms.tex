%%
%% Beginning of file 'sample61.tex'
%%
%% Modified 2016 September
%%
%% This is a sample manuscript marked up using the
%% AASTeX v6.1 LaTeX 2e macros.
%%
%% AASTeX is now based on Alexey Vikhlinin's emulateapj.cls 
%% (Copyright 2000-2015).  See the classfile for details.

%% AASTeX requires revtex4-1.cls (http://publish.aps.org/revtex4/) and
%% other external packages (latexsym, graphicx, amssymb, longtable, and epsf).
%% All of these external packages should already be present in the modern TeX 
%% distributions.  If not they can also be obtained at www.ctan.org.

%% The first piece of markup in an AASTeX v6.x document is the \documentclass
%% command. LaTeX will ignore any data that comes before this command. The 
%% documentclass can take an optional argument to modify the output style.
%% The command below calls the preprint style  which will produce a tightly 
%% typeset, one-column, single-spaced document.  It is the default and thus
%% does not need to be explicitly stated.
%%
%%
%% using aastex version 6.1
\documentclass[twocolumn]{aastex61}

%% The default is a single spaced, 10 point font, single spaced article.
%% There are 5 other style options available via an optional argument. They
%% can be envoked like this:
%%
%% \documentclass[argument]{aastex61}
%% 
%% where the arguement options are:
%%
%%  twocolumn   : two text columns, 10 point font, single spaced article.
%%                This is the most compact and represent the final published
%%                derived PDF copy of the accepted manuscript from the publisher
%%  manuscript  : one text column, 12 point font, double spaced article.
%%  preprint    : one text column, 12 point font, single spaced article.  
%%  preprint2   : two text columns, 12 point font, single spaced article.
%%  modern      : a stylish, single text column, 12 point font, article with
%% 		  wider left and right margins. This uses the Daniel
%% 		  Foreman-Mackey and David Hogg design.
%%
%% Note that you can submit to the AAS Journals in any of these 6 styles.
%%
%% There are other optional arguments one can envoke to allow other stylistic
%% actions. The available options are:
%%
%%  astrosymb    : Loads Astrosymb font and define \astrocommands. 
%%  tighten      : Makes baselineskip slightly smaller, only works with 
%%                 the twocolumn substyle.
%%  times        : uses times font instead of the default
%%  linenumbers  : turn on lineno package.
%%  trackchanges : required to see the revision mark up and print its output
%%  longauthor   : Do not use the more compressed footnote style (default) for 
%%                 the author/collaboration/affiliations. Instead print all
%%                 affiliation information after each name. Creates a much
%%                 long author list but may be desirable for short author papers
%%
%% these can be used in any combination, e.g.
%%
%% \documentclass[twocolumn,linenumbers,trackchanges]{aastex61}

%% AASTeX v6.* now includes \hyperref support. While we have built in specific
%% defaults into the classfile you can manually override them with the
%% \hypersetup command. For example,
%%
%%\hypersetup{linkcolor=red,citecolor=green,filecolor=cyan,urlcolor=magenta}
%%
%% will change the color of the internal links to red, the links to the
%% bibliography to green, the file links to cyan, and the external links to
%% magenta. Additional information on \hyperref options can be found here:
%% https://www.tug.org/applications/hyperref/manual.html#x1-40003

%% If you want to create your own macros, you can do so
%% using \newcommand. Your macros should appear before
%% the \begin{document} command.
%%
\newcommand{\vdag}{(v)^\dagger}
\newcommand\aastex{AAS\TeX}
\newcommand\latex{La\TeX}
\newcommand{\sm}{M_\odot}
\newcommand{\sr}{R_\odot}

%% Reintroduced the \received and \accepted commands from AASTeX v5.2
\received{\today}
\revised{\today}
\accepted{\today}
%% Command to document which AAS Journal the manuscript was submitted to.
%% Adds "Submitted to " the arguement.
\submitjournal{ApJ}

%% Mark up commands to limit the number of authors on the front page.
%% Note that in AASTeX v6.1 a \collaboration call (see below) counts as
%% an author in this case.
%
%\AuthorCollaborationLimit=3
%
%% Will only show Schwarz, Muench and "the AAS Journals Data Scientist 
%% collaboration" on the front page of this example manuscript.
%%
%% Note that all of the author will be shown in the published article.
%% This feature is meant to be used prior to acceptance to make the
%% front end of a long author article more manageable. Please do not use
%% this functionality for manuscripts with less than 20 authors. Conversely,
%% please do use this when the number of authors exceeds 40.
%%
%% Use \allauthors at the manuscript end to show the full author list.
%% This command should only be used with \AuthorCollaborationLimit is used.

%% The following command can be used to set the latex table counters.  It
%% is needed in this document because it uses a mix of latex tabular and
%% AASTeX deluxetables.  In general it should not be needed.
%\setcounter{table}{1}

%%%%%%%%%%%%%%%%%%%%%%%%%%%%%%%%%%%%%%%%%%%%%%%%%%%%%%%%%%%%%%%%%%%%%%%%%%%%%%%%
%%
%% The following section outlines numerous optional output that
%% can be displayed in the front matter or as running meta-data.
%%
%% If you wish, you may supply running head information, although
%% this information may be modified by the editorial offices.
\shorttitle{iPTF16abc}
\shortauthors{Cao et al.}
%%
%% You can add a light gray and diagonal water-mark to the first page 
%% with this command:
\watermark{DRAFT}
%% where "text", e.g. DRAFT, is the text to appear.  If the text is 
%% long you can control the water-mark size with:
%  \setwatermarkfontsize{dimension}
%% where dimension is any recognized LaTeX dimension, e.g. pt, in, etc.
%%
%%%%%%%%%%%%%%%%%%%%%%%%%%%%%%%%%%%%%%%%%%%%%%%%%%%%%%%%%%%%%%%%%%%%%%%%%%%%%%%%

%%%%%%%%%%%%%%%%%%%%%%%%%%%%%%%%%%%%%%%%%%%%%%%%%%%%%%%%%%%%%%%%%%%%%%%%%%%%%%%%
%%
%% The following section defines new commands for comments from co-authors
%%
\newcommand{\ycao}[1]{{\color{red} ycao: {#1}}}
%%
%%%%%%%%%%%%%%%%%%%%%%%%%%%%%%%%%%%%%%%%%%%%%%%%%%%%%%%%%%%%%%%%%%%%%%%%%%%%%%%%

%% This is the end of the preamble.  Indicate the beginning of the
%% manuscript itself with \begin{document}.

\begin{document}

\title{Optifcal Observations of an Extraordinarily Young Type Ia Supernova iPTF16abc}

%% LaTeX will automatically break titles if they run longer than
%% one line. However, you may use \\ to force a line break if
%% you desire. In v6.1 you can include a footnote in the title.

%% A significant change from earlier AASTEX versions is in the structure for 
%% calling author and affilations. The change was necessary to implement 
%% autoindexing of affilations which prior was a manual process that could 
%% easily be tedious in large author manuscripts.
%%
%% The \author command is the same as before except it now takes an optional
%% arguement which is the 16 digit ORCID. The syntax is:
%% \author[xxxx-xxxx-xxxx-xxxx]{Author Name}
%%
%% This will hyperlink the author name to the author's ORCID page. Note that
%% during compilation, LaTeX will do some limited checking of the format of
%% the ID to make sure it is valid.
%%
%% Use \affiliation for affiliation information. The old \affil is now aliased
%% to \affiliation. AASTeX v6.1 will automatically index these in the header.
%% When a duplicate is found its index will be the same as its previous entry.
%%
%% Note that \altaffilmark and \altaffiltext have been removed and thus 
%% can not be used to document secondary affiliations. If they are used latex
%% will issue a specific error message and quit. Please use multiple 
%% \affiliation calls for to document more than one affiliation.
%%
%% The new \altaffiliation can be used to indicate some secondary information
%% such as fellowships. This command produces a non-numeric footnote that is
%% set away from the numeric \affiliation footnotes.  NOTE that if an
%% \altaffiliation command is used it must come BEFORE the \affiliation call,
%% right after the \author command, in order to place the footnotes in
%% the proper location.
%%
%% Use \email to set provide email addresses. Each \email will appear on its
%% own line so you can put multiple email address in one \email call. A new
%% \correspondingauthor command is available in V6.1 to identify the
%% corresponding author of the manuscript. It is the author's responsibility
%% to make sure this name is also in the author list.
%%
%% While authors can be grouped inside the same \author and \affiliation
%% commands it is better to have a single author for each. This allows for
%% one to exploit all the new benefits and should make book-keeping easier.
%%
%% If done correctly the peer review system will be able to
%% automatically put the author and affiliation information from the manuscript
%% and save the corresponding author the trouble of entering it by hand.

\correspondingauthor{Yi Cao}
\email{ycao16@uw.edu}

\author[0000-0002-8036-8491]{Yi Cao}
\affil{eScience Institute and Astronomy Department, University of Washington,
  Seattle, WA 98195}

\author{Friends}
\affil{the intermediate Palomar Transient Factory}

%% Note that the \and command from previous versions of AASTeX is now
%% depreciated in this version as it is no longer necessary. AASTeX 
%% automatically takes care of all commas and "and"s between authors names.

%% AASTeX 6.1 has the new \collaboration and \nocollaboration commands to
%% provide the collaboration status of a group of authors. These commands 
%% can be used either before or after the list of corresponding authors. The
%% argument for \collaboration is the collaboration identifier. Authors are
%% encouraged to surround collaboration identifiers with ()s. The 
%% \nocollaboration command takes no argument and exists to indicate that
%% the nearby authors are not part of surrounding collaborations.

%% Mark off the abstract in the ``abstract'' environment. 
\begin{abstract}

  In this paper, we present observations of a young normal Type Ia supernova
  iPTF16abc. Our analysis shows that: blabla ...

\end{abstract}

%% Keywords should appear after the \end{abstract} command. 
%% See the online documentation for the full list of available subject
%% keywords and the rules for their use.
\keywords{methods: observational --- supernovae: individual (iPTF16abc)}

%% From the front matter, we move on to the body of the paper.
%% Sections are demarcated by \section and \subsection, respectively.
%% Observe the use of the LaTeX \label
%% command after the \subsection to give a symbolic KEY to the
%% subsection for cross-referencing in a \ref command.
%% You can use LaTeX's \ref and \label commands to keep track of
%% cross-references to sections, equations, tables, and figures.
%% That way, if you change the order of any elements, LaTeX will
%% automatically renumber them.

%% We recommend that authors also use the natbib \citep
%% and \citet commands to identify citations.  The citations are
%% tied to the reference list via symbolic KEYs. The KEY corresponds
%% to the KEY in the \bibitem in the reference list below. 

\section{Introduction}
\label{sec:intro}

Although Type Ia supernovae (SNe Ia) have been extensively used as
standardizable candles, their progenitor scenarios and explosion
physics are still in debate (see a recent review by
\citealt{2014ARA&A..52..107M}). Detailed extermely early-phase
observations are one of the most promising avenues to further
constrain this problem.

While the shock breakout of a SN Ia occurs on a sub-second timescale,
the subsequent quasi-adiabatic expanding and cooling of the unbinded
ejecta produces thermal emissions that can be used to infer the
original size of the exploding star
\citep{2010ApJ...708..598P,2011ApJ...728...63R}. Comparing models of
these cooling emissions to the earliest-phase data of SN2011fe,
\citet{2012ApJ...744L..17B} concluded that the radius of the
progenitor star is $\lesssim0.01\sr$ where $\sr$ is the solar
radius. Combining this size constraint and the measured ejecta mass to
derive the mean density of the progenitor star, we confirmed that the
progenitor star is compact and degenerate. Admittedly, due to the
initial small surface area of the progenitor star, the shock cooling
emission of a SN Ia decays drastically as the ejecta expands. Given
typical parameters of a SN Ia, this thermal emission is visible from
events up to $\sim 10\,\textrm{Mpc}$ within one day of their explosions.

Another expectation from the extremely early-phase observations of a
SN Ia is the excess emission from collisions between SN ejecta and a
companion star, a natural consequence from the single-degenerate
progenitor hypothesis \citep{1973ApJ...186.1007W,2010ApJ...708.1025K}.
In a low-velocity SN Ia iPTF14atg, \citet{2015Natur.521..328C} for the
first time detected a strong and declining ultraviolet pulse within a
few days of the SN explosion which is best interpreted as the
SN-companion collision. This signature has been searched in a number
of nearby, early-phase normal SNe Ia, but most of these studies result
in no detection
\citep{2010ApJ...722.1691H,2011ApJ...741...20B,2012ApJ...744...38F,
  2012ApJ...744L..17B,2015Natur.521..332O,
  2013ApJ...778L..15Z,2015ApJ...799..106G,2016ApJ...826..144S,
  2015ApJS..221...22I}. The exception is SN2012cg which
\citet{2016ApJ...820...92M} claimed detection of blue excess in the
earliest-phase light curve and attribute it to SN-companion collision.
However, this statement is recently challenged by
\citet{2016arXiv161007601S}. In fact, it is not surprising that no
SN-companion collision has been observationally confirmed, because
only up to $\sim 10\%$ of events from the single-degenerate channel
have the preferred binary geometry for us to see the collision
signatures.

More commonly, the only observed light curve of a SN Ia is purely
powered by the radioactive decay of synthesized $^{56}$Ni. Since
$^{56}$Ni atoms are synthesized, mixed and deposited into different
layers inside the ejecta, the SN may experience a dark period after
the SN shock breakout and before the radioactive energy diffuses to
the photosphere \citep{2014ApJ...784...85P}. In the case of strong
mixing, $^{56}$Ni energy can reach the photosphere rapidly. As such,
the dark period is very short and the light curve shows a fast initial
rise. In the opposite case, it may take up to a couple of days for
the $^{56}$Ni energy to diffuse to the photosphere. The initial rise
of the light curve is then moderate \citep{2016ApJ...826...96P}. To
summarize, the initial rise of the light curve of a SN Ia conveys
information on distribution of synthesized $^{56}$Ni. 

In this paper, we report observations of an extraordinarily young SN
Ia iPTF16abc, which was discovered by the intermediate Palomar
Transient Factory on 2016 April $3.36$\footnote{all times in this
  paper are in UTC.} at $\textrm{R.A.}=13^h34^m45.49^s$,
$\textrm{Dec.}=+13^d51^m14.3^s$ (J2000) with a $g$-band magnitude of
$21.31\pm0.27$ \citep{2016PASP..128k4502C,2016ATel.8907....1M}. The
transient is spatially coincident with a tidal tail of the galaxy
NGC\,5221 at 100\,Mpc. No activity was detected at the same location
down to $g=22.1$\,mag on April $2.42$. Our spectroscopic follow-up
compaign classified iPTF16abc as a normal SN Ia
\citep{2016ATel.8909....1C}.

This paper is organized as follows: Section \ref{sec:obs} describes
photometric and spectroscopic observations of iPTF16abc. Section
\ref{sec:usual_staff} establishes that iPTF16abc is a normal SN Ia
in NGC\,5221. Section \ref{sec:first_light} analyzes the early
light curve and spectra.

\section{Observations}
\label{sec:obs}

\ycao{TODO: include LCOGT data in both text and figures.}

\begin{figure*}[htb]
  \centering
  \includegraphics[width=0.95\textwidth]{lightcurve.pdf}
  \caption{Multi-band light curves of iPTF16abc are shown. Filters are
    denoted by different colors and observation instruments by
    different markers. The $t_{max}$ time is the B-band maximum
    determined by SALT2 (Section). The black ticks near the top of the
    figure shows epochs of spectroscopic observations.}
  \label{fig:lightcurve}
\end{figure*}

As part of the iPTF transient survey in the 2016 spring quarter, the
field of iPTF16abc was observed in $g$- or $R$-band every night by the
CFH12K camera \citep{2000SPIE.3965...58S} on the 48-inch telescope at
Palomar Observatory (P48). The images were processed by the IPAC image
subtraction and discovery pipeline which subtracts off the background
galaxy light with stacked pre-SN images and performs forced
point-spread-function (PSF) photometry at the location of the SN. The
photometry is then calibrated to the PTF photometric catalog
\citep{2012PASP..124..854O}.

After discovery, we utilized the rainbow camera of the SED
Machine (\ycao{REF}) mounted on the 60-inch telescope at Palomar
Observatory (P60) to undertake photometric observations in $g$, $r$
and $i$ filters. The image differencing against the archival SDSS
images and forced PSF photometry on the subtracted images were
performed by the Fremling Automated Pipeline
\citep{2016A&A...593A..68F}. The photometry is also calibrated to the
SDSS catalog.

Las Cambres Observatory Global Network (LCOGT) also carry out photometric
observations in the \textit{BVgri} filters with its 1-m telescope network.
\ycao{Some info about data reduction}

In space, \textit{Swift} observed iPTF16abc for 14 epochs, covering
from the very early phase to the post-peak phase. Aperture photometry
are carried out on the images taken by its Ultraviolet-Optical
Telescope (UVOT) with the usual procedures in the HEASoft and
corrected for the coincident loss and aperture loss. No pre-SN UVOT
image at the SN location is available in the \textit{Swift} archive.
Visual inspection to the UVOT images suggests that the background
galaxy light in the UVOT filters is probably negligible. No X-ray
emission was detected at the location of the SN by the X-ray Telescope
(XRT) in any of these for epochs.

The multi-color light curves of iPTF16abc are illustrated in Figure
\ref{fig:lightcurve}.  For convenience, all magnitudes are in the AB
system with a zero point of $3631$\,Jy in all filters.

Spectroscopic observations of iPTF16abc were undertaken with the
Gemini Multi-Object Spectrograph (GMOS; \citealt{2004PASP..116..425H})
on the Gemini North telescope, Low-Resolution Imaging Spectrometer
(LRIS; \citealt{1995PASP..107..375O}) on the Keck-I telescope, DEep
Imaging Multi-Object Spectrograph (DEIMOS;
\citealt{2003SPIE.4841.1657F}) on the Keck-II telescope, The Andalucia
Faint Object Spectrograph and Camera (ALFOSC\footnote{ALFOSC
  instrument webpage:
  \url{http://www.not.iac.es/instruments/alfosc/}}) on the Nordic
Optical Telescope (NOT), and X-shooter \citep{2011A&A...536A.105V} and
Ultraviolet and Visual Echelle Spectrograph (UVES;
\citealt{2000SPIE.4008..534D}) on the Very Large Telescope (VLT). The
observing log is listed in Table \ref{tab:spec_obs_log} and the
low-resolution spectral sequence is shown in Figure
\ref{fig:spec_seq}.

% currently the following table and figure do not include data from LCOGT
\begin{deluxetable*}{cccccc}
  \tablecaption{Spectroscopic observations of iPTF16abc \label{tab:spec_obs_log}}
  \tablehead{
    \colhead{Observation Date} & \colhead{SN phase} & \colhead{Telescope/Instrument} &
    \colhead{Exposure Time (s)} & \colhead{Wavelength Coverage (\AA)} & \colhead{Resolution}
  }
  \startdata
  2016 April $05.88$ & $-15.8$ & Gemini-North/GMOS &     & $3500$ -- $9500$  & $1900$ \\
  2016 April $06.51$ & $-15.1$ & Keck-II/DEIMOS & $1491$ & $5500$ -- $8100$  & $2000$ \\
  2016 April $08.51$ & $-13.1$ & Keck-II/DEIMOS & $900$  & $5500$ -- $8100$  & $2000$ \\
  2016 April $10.38$ & $-11.3$ & Keck-I/LRIS    & $300$  & $3000$ -- $10000$ & $1000$ \\
  2016 April $14.20$ & $-7.5$  & VLT/XSHOOTER   &        & $3000$ -- $25000$ & $10000$ \\
  2016 April $16.??$ & $-5.?$  & VLT/UVES       &        & covers \ion{Ca}{2}\,H$+$K and \ion{Na}{1}\,D lines & $40000$ \\
  2016 April $28.??$ & $6.4$   & NOT/ALFOSC     &        & $3300$ -- $9000$  & $360$ \\
  2016 May   $10.42$ & $18.8$  & Keck-I/LRIS    & $600$  & $3000$ -- $10000$ & $1000$ \\
  2016 May   $12.03$ & $20.4$  & VLT/XSHOOTER   &        & $3000$ -- $25000$ & $10000$ \\
  \enddata
\end{deluxetable*}

\begin{figure*}[!htb]
  \centering
  \includegraphics[width=0.9\textwidth]{spec_sequence.pdf}
  \caption{Low-resolution spectra of iPTF16abc are shown in the
    chronical sequence. The phases in units of days are labeled next
    to corresponding spectra. Telluric absorption bands are grayed
    out.}
  \label{fig:spec_seq}
\end{figure*}


\section{Reddening, Classification and Host Galaxy}
\label{sec:usual_staff}

\subsection{Reddening}
\label{sec:reddening}

The foreground Galactic extinctioin along the directionof iPTF16abc
has $E(B-V)=0.0279\,\textrm{mag}$ \citep{2011ApJ...737..103S}.

In the highest-resolution spectrum of iPTF16abc by UVES, individual
components of both \ion{Ca}{2}\,H$+$K and \ion{Na}{1}\,D doublets show
double-absorption profiles (Figure \ref{fig:narrow_features}),
indicating two sources of absorption along the line of sight. Fitting
two Gaussian kernels to each line of the \ion{Na}{1} doublet
simultaneously leads to redshifts of $0.02313820\pm0.00000032$ and
$0.02322408\pm0.00000033$. The total equivalent widths of the
\ion{Na}{1}\,D1 and D2 lines are $0.595\pm0.009\,\textrm{\AA}$ and
$0.609\pm0.008\,\textrm{\AA}$m, respectively. Using the empirical
relation between the equivalent width of \ion{Na}{1}\,D lines and
reddening $E(B-V)$ \citep{2012MNRAS.426.1465P}, we derive
$E(B-V)=0.361\pm0.025\textrm{mag}$.

In the X-shooter spectra of iPTF16abc, we also identify narrow
absorption features \ion{K}{1}\,7665\,\AA and 7699\,\AA at consistent
redshifts. However, the resolution of the X-shooter spectra is not
fine enough to resolve the double-absorption profiles of \ion{Ca}{2}
and \ion{Na}{1}. The X-shooter spectra do not show the diffusive
interstellar band at 5780\,\AA as well.  Albeit some debates (e.g.,
\citealt{2013ApJ...779...38P}), the $E(B-V)$ value derived from
\ion{Na}{1}\,D absorption provides the best estimate in the case of
iPTF16abc.

\begin{figure}[htb]
  \centering
  \includegraphics[width=0.45\textwidth]{narrow_abs_features.pdf}
  \caption{Narrow absorption lines of iPTF16abc are shown in this
    figure. The zero velocity corresponds to the redshift of the
    apparent host NGC\,5221.}
  \label{fig:narrow_features}
\end{figure}

The \ion{Na}{1}\,D doublet are seen in multiple spectra spanning
from pre-peak to post-peak phases. Despite the instrumental widening of
different instrument configurations, we do not detect obvious variation
in the profiles of the doublet.


\subsection{Classfication}
\label{sec:classification}

We run Supernova Identification (SNID; \citealt{2007ApJ...666.1024B})
on the low-resolution spectrum of iPTF16abc at $+18.8$ and found best
matches to normal SNe Ia. In fact, the characteristic features of a SN
Ia, such as \ion{Si}{2}, \ion{S}{2}, can be easily identified in the
spectra of iPTF16abc (Figure \ref{fig:spec_seq}).

Next we standardize the light curve of iPTF16abc by feeding its P48
and P60 light curves into the \texttt{sncosmo} module\footnote{The
  \texttt{sncosmo} Python module is available at
  \url{https://sncosmo.readthedocs.io/en/v1.4.x/}.} and fit a light
curve model, which composes of the SALT2 template
\citep{2007A&A...466...11G} modified by the line-of-sight extinction
curve \citep{1999PASP..111...63F} with $E(B-V)$ values from Section
\ref{sec:reddening} and $R_V=3.1$. We obtain the rest-frame
\textit{B}-band peak time $\textrm{MJD}_{max}=57499.65\pm0.02$, the coefficient
of the zeroth principle component $x_0=0.0275\pm0.0002$, the
coefficient of the first principle component $x_1=1.200\pm0.043$, and
the color term $c=-0.3353\pm0.0054$. The best-fit model also gives an
unreddened apparent peak magnitude of $m^*_{B}=14.4\,\textrm{mag}$ in
the SN rest frame.

For convenience, in the following sections, we define the best-fit
value $\textrm{MJD}_{max}=57499.65$ as phase $t=0$.

\subsection{Host Galaxy}
\label{sec:host}

After establishing iPTF16abc as a normal SN ia, we use the latest
calibration \citep{2014A&A...568A..22B} of the Phillips relation
\citep{1993ApJ...413L.105P} using $m^*_{B}$, $x_1$ and $c$ to derive a
distance modulus $\mu=34.66\pm0.03\,\textrm{mag}$ to the SN, provided
that the host galaxy of iPTF16abc has a stellar mass less than
$10^{10}\sm$.

The location of iPTF16abc is spatially coincident with a tidal tail of
galaxy NGC\,5221. \citet{2007A&A...465...71T} derived a distance
modulus of $35.0\pm0.4\,\textrm{mag}$ from the Tully-Fisher
relation. This distance modulus is consistent with that of iPTF16abc.

Separately, \citet{1998A&AS..130..333T} observed the 21-cm line in
this galaxy and measured a redshift of $0.0233303\pm0.000027$.  The
two components in the \ion{Na}{1}\,D have a relative velocity of
$-57.6\pm8.1\,\textrm{km}\,\textrm{s}^{-1}$ and
$-31.8\pm8.1\,\textrm{km}\,\textrm{s}^{-1}$, suggesting that both
absorption resources are probaly located on the tidal tail of
NGC\,5221.


\section{First Light And Explosion Time}
\label{sec:first_light}

In this section, we estimate the first light and explosion time of
iPTF16abc.

\subsection{Light Curve Fit}
\label{sec:lc_fit}

Extrapolation of the earliest-phase light curve with a simple model is
usually used to estimate the first light time. In theory,
\citet{1982ApJ...253..785A} derived a quadratic law for an expanding
fireball with a constant temperature. In order to account for the
temperature change, here we utilize a power-law model
\begin{equation}
  \label{eq:broken_power_law}
  f(t) \left\{
    \begin{array}{ll}
      = 0,\ \textrm{when}\ t<t_0 \\
      \propto (t-t_0)^{\alpha},\ \textrm{when}\ t>t_0
    \end{array}
  \right.\ .
\end{equation}
Since the first few detections of the SN were made in the \textit{g}
band, we perform the fit only to the \textit{g}-band light curve. We
experiment the fitting procedure with different time windows and find
that the SN flux between $t=-19\,\textrm{days}$ and
$t=-15\,\textrm{days}$ rises approximately linearly. The joint
distribution of $\alpha$ and $t_0$ for this time window is given in
the left panel of Figure \ref{fig:early_lc_fit} and the best fit model
with $\alpha = 0.94$ and $t_0=-18.47$ is shown in the right
panel. With this best-fit value of $t_0$, our first observation was
made only $0.18\,\textrm{day}$ after the first light of the SN. The
total rise time to the \textit{B}-band peak is $18.47$ days.

\begin{figure*}[htb]
  \centering
  \includegraphics[width=0.45\textwidth]{rise_time_power_law_index.pdf}
  \includegraphics[width=0.45\textwidth]{early_lc.pdf}
  \caption{Broken Power low fitting to the early $g$-band light
    curve.
    \textit{Left:} the joint distribution of $t_0$ and
    $\alpha$. The cross marker denotes the best-fit parameters. The
    inner and outer contours represent the $68\%$ and $99.7\%$
    confidence levels.
    \textit{Right:} the best-fit model of $\alpha=0.94$ and
    $t_0=-18.47\,\textrm{days}$ is illustrated against the
    data.
  }
  \label{fig:early_lc_fit}
\end{figure*}

Since $t=-15\,\textrm{days}$, the \textit{g}-band light curve rises
significantly faster than the best-fit model above, indicating a
greater value of the power-law index $\alpha$. In fact, the
light curve between $t=-14$ and $t=-8$\,days can be fitted with a
power law of index $1.40$.

A more sophisticated broken power-law model has also been used to fit
SN early-phase light curves (e.g.,
\citealt{2013ApJ...778L..15Z,2014ApJ...783L..24Z,
  2016arXiv161202097Z,2016arXiv161202097Z}). Our above result has
indicates that the \textit{g}-band light curve of iPTF16abc can also
be fitted by a broken-power law with a power index changing from
$0.94$ to $1.40$. In fact, many young SNe, including SN2011fe
\cite{2016arXiv161202097Z} and iPTF16abc, have early light curves that
follow broken power-laws \citep{2016arXiv161202725Z}). Our above
result has indicates that the \textit{g}-band light curve of iPTF16abc
can also be fitted by a broken-power law with a power index changing
from $0.94$ to $1.40$.

The initial rise of a SN depends on the distribution of $^{56}$Ni
in the ejecta. Comparing the rapid initial rise of iPTF16abc to
theoretical calculations in \citet{2016ApJ...826...96P} suggests
strong mixing of $^{56}$Ni in the iPTF16abc ejecta. The radioactive
energy from $^{56}$Ni can quickly diffuse to the SN photosphere and
power its light curve after the SN explosion. 

\subsection{Expansion Velocity Fit}
\label{sec:early_vel}

The first light of a SN is not necessarily its explotion time
$t_{exp}$.  Thanks to different deposition of $^{56}$Ni in the SN
ejecta, as pointed out in \citet{2014ApJ...784...85P}, the SN may
experience a dark period before the radioactive energy from the
shallowest $^{56}$Ni layer escapes the ejecta. Therefore, 
\citet{2014ApJ...784...85P} suggests to examine evolution of absorption
line velocities, because the velocities do not depend on the $^{56}$Ni
mixing and are expected to evolve roughly as $v\propto(t-t_{exp})^{-0.22}$. 

We focus on measuring velocities of the strongest \ion{Si}{2}\,6355
line, as the second most commonly used \ion{Ca}{2} IR triplet is very
weak in iPTF16abc. Visual inspection of the early-phase spectra shows
no sign of any high-velocity component of the \ion{Si}{2} line and
that the red wing of \ion{Si}{2} overlaps the \ion{C}{2}\,6580
line. Therefore, we fit simultaneously two gaussian kernels for the
two lines and a linear term to account for the continuum component.
Then the velocities are calculated at the minimum of the \ion{Si}{2}
gaussian kernel.

\begin{figure*}[!thb]
  \centering
  \includegraphics[width=0.45\textwidth]{exp_date_chi2.pdf}
  \includegraphics[width=0.45\textwidth]{SiIIVelocity.pdf}
  \caption{Constraints on $t_{exp}$ from fitting the velocity
    evolution of \ion{Si}{2}. \textit{Left panel:} the dashed, solid
    and dash-dotted curves show $\chi^2$ for fitting power laws with
    indices $-0.20$, $-0.22$ and $-0.24$, respectively. The red
    vertical line and the orange region indicate $t_0$ and its
    3-$\sigma$ confidence interval from Section
    \ref{sec:lc_fit}. \textit{Right panel:} Observed \ion{Si}{2}\,6355
    velocities with the best-fit power-law velocity with an index of
    $-0.22$.}
  \label{fig:velocity_t_exp}
\end{figure*}

Fitting the measured velocities of \ion{Si}{2}\,6355 line to the
$v\propto(t-t_{exp})^{-0.22}$ model leads to the best-fit explosion
time $t_{exp}=-17.95\,\textrm{days}$ with a 3-$\sigma$ confidence
interval between $-17.4\,\textrm{days}$ and $-18.3\,\textrm{days}$
(Figure \ref{fig:velocity_t_exp}). We further test the robustness of
a fixed power-law index by altering the index to $-0.20$ and $-0.24$
and find consistent results within respective 3-$\sigma$ confidence
intervals.

The derived $t_{exp}$ is roughly consistent to $t_0$ estimated in
Section \ref{sec:lc_fit}, confirming that the dark period of iPTF16abc
is very short. This conclusion is consistent with the inference from the
rapid initial rise of iPTF16abc in Section \ref{sec:lc_fit}. 

Among the several SNe with measurements of $t_{exp}$ and $t_0$
\citep{2014ApJ...784...85P,2016ApJ...826..144S}, iPTF16abc has the
shortest dark period and longest rise time ($18.47$ days). In
comparison, SN2011fe is inferred to have a dark period of $\sim 1$
day \citep{2014ApJ...784...85P} and a rise time of $17.7$ days
\citep{2013A&A...554A..27P}; ASASSN-14lp has a dark period of a couple
of days and a rise time of $16.94$ days. This seems to suggest that
the time difference between SN explosion and the \textit{B}-band maximum,
which equals to summation of the dark period and the rise time, is a
constant for a SN Ia. This is perhaps because the \textit{B}-band maximum
time only depends on the total amount of $^{56}$Ni. 

\subsection{Strong and Short-Lived Carbon Features}
\label{sec:carbon}

Interestingly, the first few spectra of iPTF16abc show clear detection
of \ion{C}{2}\,6580 and \ion{C}{2}\,7234 lines. Their velocity
evolution is shown in Figure \ref{fig:velocity_t_exp} as well.

The \ion{C}{2} lines only appear in the very early phases. The
equivalent widths of detected \ion{C}{2}\,6580 and \ion{C}{2}\,7234
lines are shown in Figure \ref{fig:ew}, and are compared to that of
the \ion{Si}{2}\,6355 line. Especially in the first spectrum taken at
$t=-15.8$ days, the \ion{C}{2} equivalent widths are stronger than
that of \ion{Si}{2}.

\begin{figure}[!htb]
  \centering
  \includegraphics[width=0.45\textwidth]{EW.pdf}
  \caption{Equivalent Width of \ion{Si}{2}\,6355, \ion{C}{2}\,6580 and
    \ion{C}{2}\,7234 lines}
  \label{fig:ew}
\end{figure}

The strong and short-lived \ion{C}{2} features suggest that
the outermost layers of the ejecta contain a certain amount of
carbon which is responsible for the \ion{C}{2} absorption. The
abundance of carbon drops dramatically into the inner layers
of the ejecta.

Carbon signatures are seen in over $1/4$ of all normal SNe Ia before
maxima
\citep{2011ApJ...732...30P,2012MNRAS.425.1917S,2011ApJ...743...27T},
but the signatures are usually weak. Even in SN2011fe, the \ion{C}{2}
features are not strong in the first spectra
\citep{2012ApJ...752L..26P}.  The only normal SN Ia known to have
strong \ion{C}{2} features at very early phases is SN2013dy
\citep{2013ApJ...778L..15Z}. Unlike iPTF16abc, the equivalent widths
of \ion{C}{2} features in SN2013dy are as large as that of
\ion{Si}{2}\,6355.


\subsection{Discussions and Physical Constraints}
\label{sec:lc_energy}

The early radiation of a SN Ia may have multiple resources: SN
shockbreakout, SN-companion collision, and radioactive activity. Since
each provides interesting constraints on the progenitor properties, we
explore all the three posibilities.

\subsubsection{SN Shock Breakout}

The shock breakout of a SN Ia lasts for a fraction of a second due to
the small size of the exploding star. However, the subsequent cooling
phase may last longer (e.g., \citealt{2010ApJ...708..598P}).
Following the analysis of SN2011fe in \citet{2012ApJ...744L..17B}, we
compare the early-phase \textit{g}-band light curve of iPTF16abc with
two cooling models \citep{2011ApJ...728...63R, 2010ApJ...708..598P}
and reach a not very constraining conclusion that the radius of the
progenitor star of iPTF16abc should be $<1\sr$. In fact, the cooling
emission is negligible at the time of the first detection of
iPTF16abc.

\subsubsection{SN companion}

If iPTF16abc is born in a single-degenerate channel, with an odd of
$\lesssim10\%$, we expect to see emission produced by SN ejecta
slamming into the companion. However, according to calculations by
\citet{2010ApJ...708.1025K}, collision between SN ejecta and the
companion star produces thermal emission with a spectrum that peaks
in the ultraviolet, and the Rayleigh-Jeans tail emission in the \textit{g}
band is very weak. 

In the case of SN2012cg, in order to illustrate a blue excess in the
early light curve, \citet{2016ApJ...820...92M} fitted a power-law
model to its light curves between $t=-14$ days and $t=-8$ days and
subtracted the extrapolated fluxes at early epochs. Then these authors
attribute the excess to SN-companion collision. In our iPTF16abc, we
follow the same procedure and also find a similar excess
(Figure \ref{fig:against_sn2012cg}).

\begin{figure}[!htb]
  \centering
  \includegraphics[width=0.45\textwidth]{another_early_lc.pdf}
  \caption{The early-phase \textit{g}-band light curve of iPTF16abc is
    compared against the best-fit power-law model to the data segment
    between $t=-14$ and $t=-8$ days. Note the ``excess'' between the
    data and the model during $t=-18$ and $t=-15$ days}
  \label{fig:against_sn2012cg}
\end{figure}

Actually, for any early light curve that can be approximated by a
broken power law, as long as the asymptotic power-law index near $t=0$
is less than that at late time, one can always find such an excess
near $t=0$. There are not many SNe that were observed at
extraordinarily early phases. The odd that both SN2012cg and iPTF16abc
show SN-companion collision signautres is very small.

Separately, our first \textit{Swift} observation of iPTF16abc was made
about two days after explosion. However, the ultraviolet light curve of
iPTF16abc does not show any excess. 

\subsubsection{Explosion Mechanism}

Our extraordinarily early observations of iPTF16abc put two constraints
on the mechanism of the SN explosion: first, during the SN explosion,
synthesized $^{56}$Ni should be strongly mixed in the ejecta; second,
unburned carbon atoms are accelerated to very high velocities in the ejecta.
In addition, the commonly seen \ion{Ca}{2} IR triplet is weak in iPTF16abc,
implying a little amount of calsium in the ejecta. 

\section{Conclusion}
\label{sec:conclusion}

In this paper, we present observations of a normal Type Ia supernova
iPTF16abc, which is located at one tidal tail of NGC\,5221. Our extraordinarily
early-phase optical observations of this supernova show that:
\begin{itemize}
\item Extrapolation of the early light curve shows that our first observation
  was made only $0.18$ days after the first light of the SN.
\item The rapid initial rise and analysis of the line velocity both shows that
  the dark period of iPTF16abc is very short. Hence, the first light time of
  the SN is a good approximation to the time of the SN explosion.
\item Strong and short-lived carbon features in the early spectra of iPTF16abc
  indicates concentration of unburned carbon in the fastest-moving part of the
  ejecta.
\end{itemize}

\acknowledgements

\bibliographystyle{aasjournal}
\bibliography{ref}

\end{document}
